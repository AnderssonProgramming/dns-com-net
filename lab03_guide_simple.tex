% Simplified version for online compilation
\documentclass[12pt,a4paper]{article}
\usepackage[utf8]{inputenc}
\usepackage[english]{babel}
\usepackage{graphicx}
\usepackage{geometry}
\usepackage{listings}
\usepackage{xcolor}

\geometry{margin=1in}

% Basic code listing settings
\lstset{
    basicstyle=\ttfamily\footnotesize,
    keywordstyle=\color{blue},
    commentstyle=\color{green!60!black},
    frame=single,
    breaklines=true
}

\title{\textbf{Laboratory No. 3}\\
\textbf{DNS Server Configuration}\\
\large Computer Networks}

\author{Student 1: Cristian\\
Student 2: Andersson\\
\\
Computer Networks Course\\
University Name}

\date{\today}

\begin{document}

\maketitle
\newpage

\tableofcontents
\newpage

\section{Objectives}

\subsection{General Objective}
Implement and configure a complete DNS infrastructure using multiple operating systems to understand name resolution principles, zone transfers, and distributed DNS services.

\subsection{Specific Objectives}
\begin{enumerate}
    \item Configure primary DNS servers on Slackware Linux and Solaris
    \item Implement secondary DNS servers for redundancy
    \item Set up DNS zone files with A, AAAA, CNAME, and NS records
    \item Test DNS resolution functionality using nslookup
    \item Document configuration and troubleshooting procedures
\end{enumerate}

\section{Implementation}

\subsection{Network Configuration}
\begin{itemize}
    \item IP Range: 10.2.77.176/24
    \item Slackware DNS: 10.2.77.176 (Primary for andersson.org.uk)
    \item Solaris DNS: 10.2.77.178 (Primary for cristian.com.it)
\end{itemize}

\subsection{Slackware DNS Configuration}

The Slackware system serves as primary DNS for andersson.org.uk domain.

\subsubsection{Main Configuration File}
File: /etc/named.conf

\begin{lstlisting}[language=bash]
options {
    directory "/var/named";
    allow-query { any; };
    recursion yes;
};

zone "." IN {
    type hint;
    file "caching-example/named.root";
};

zone "andersson.org.uk" IN {
    type master;
    file "andersson.org.uk.zone";
    allow-transfer { 10.2.77.178; };
};

zone "cristian.com.it" IN {
    type slave;
    file "cristian.com.it.slave";
    masters { 10.2.77.178; };
};
\end{lstlisting}

\subsubsection{Zone File Configuration}
File: /var/named/andersson.org.uk.zone

\begin{lstlisting}[language=bash]
$TTL 86400
@ IN SOA dns.andersson.org.uk. admin.andersson.org.uk. (
    2024120301  ; Serial
    3600        ; Refresh
    1800        ; Retry
    604800      ; Expire
    86400       ; Minimum TTL
)

; Name Server
@ IN NS dns.andersson.org.uk.

; IPv4 addresses
dns     IN A 10.2.77.176
server1 IN A 10.2.77.177
server2 IN A 10.2.77.178
server3 IN A 10.2.77.179

; IPv6 addresses
server1 IN AAAA 2001:db8::1
server2 IN AAAA 2001:db8::2

; Aliases
www  IN CNAME server1.andersson.org.uk.
mail IN CNAME server2.andersson.org.uk.
web  IN CNAME server1.andersson.org.uk.
\end{lstlisting}

\subsection{Solaris DNS Configuration}

The Solaris system serves as primary DNS for cristian.com.it domain.

\subsubsection{Main Configuration}
File: /etc/named.conf

\begin{lstlisting}[language=bash]
options {
    directory "/var/named";
    allow-query { any; };
    recursion yes;
};

zone "." IN {
    type hint;
    file "named.ca";
};

zone "cristian.com.it" IN {
    type master;
    file "cristian.com.it.zone";
    allow-transfer { 10.2.77.176; };
};

zone "andersson.org.uk" IN {
    type slave;
    file "andersson.org.uk.slave";
    masters { 10.2.77.176; };
};
\end{lstlisting}

\subsubsection{Zone File}
File: /var/named/cristian.com.it.zone

\begin{lstlisting}[language=bash]
$TTL 86400
@ IN SOA dns.cristian.com.it. admin.cristian.com.it. (
    2024120401  ; Serial
    3600        ; Refresh
    1800        ; Retry
    604800      ; Expire
    86400       ; Minimum TTL
)

; Name Server
@ IN NS dns.cristian.com.it.

; IPv4 addresses
dns     IN A 10.2.77.178
server1 IN A 10.2.77.180
server2 IN A 10.2.77.181
server3 IN A 10.2.77.182

; IPv6 addresses
server1 IN AAAA 2001:db8:1::1
server2 IN AAAA 2001:db8:1::2

; Aliases
www  IN CNAME server1.cristian.com.it.
mail IN CNAME server2.cristian.com.it.
web  IN CNAME server1.cristian.com.it.
ftp  IN CNAME server3.cristian.com.it.
\end{lstlisting}

\section{Testing and Verification}

\subsection{Service Startup}
\begin{lstlisting}[language=bash]
# Start DNS service
/usr/sbin/named

# Verify service is running
ps aux | grep named
netstat -ln | grep :53
\end{lstlisting}

\subsection{DNS Resolution Testing}
\begin{lstlisting}[language=bash]
# Test local domain resolution
nslookup dns.cristian.com.it
nslookup www.andersson.org.uk
nslookup server1.cristian.com.it

# Test external resolution
nslookup www.google.com
nslookup www.escuelaing.edu.co
\end{lstlisting}

\subsection{Advanced nslookup Testing}
\begin{lstlisting}[language=bash]
nslookup
> set type=A
> server1.cristian.com.it
> set type=AAAA
> server1.cristian.com.it
> set type=NS
> cristian.com.it
> set type=CNAME
> www.cristian.com.it
> exit
\end{lstlisting}

\section{Results}

The DNS implementation achieved:
\begin{itemize}
    \item Primary DNS servers configured on both Slackware and Solaris
    \item Secondary DNS replication working between servers
    \item All record types (A, AAAA, CNAME, NS) resolving correctly
    \item External domain resolution maintained
    \item Zone transfers functioning properly
\end{itemize}

\section{Troubleshooting}

Common issues encountered:
\begin{enumerate}
    \item Configuration syntax errors - Use named-checkconf
    \item Zone file format errors - Use named-checkzone
    \item Service startup problems - Check port 53 availability
    \item Resolution failures - Verify /etc/resolv.conf
\end{enumerate}

\section{Conclusions}

This laboratory provided comprehensive experience with:
\begin{itemize}
    \item Multi-platform DNS server configuration
    \item Zone file management and record types
    \item Primary/secondary DNS relationships
    \item DNS troubleshooting methodologies
    \item Network service integration
\end{itemize}

The implementation successfully created a robust DNS infrastructure demonstrating enterprise-level name resolution services across heterogeneous operating systems.

\end{document}
